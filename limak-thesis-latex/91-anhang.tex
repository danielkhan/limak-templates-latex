% !TeX root = main-limak-thesis.tex

%%%%%%%%%%%%%%%%%%%%%%%%%%%%%%%%%%%%%%%%%%%%%%%%%%%%%%%%%%%%%%%%%%%%%%%%%%%%%%%%
%% ANHANG
%%
%% Gemäß LIMAK Leitfaden: Ergänzende Materialien zur Hauptarbeit
%%
%% Typische Inhalte:
%% - Interviewleitfäden und Fragebögen
%% - Detaillierte Datenauswertungen
%% - Prozessdiagramme und technische Dokumentation
%% - Screenshots und Abbildungen
%% - Transkripte (bei Bedarf)
%% - Glossar und Abkürzungsverzeichnis
%%
%% Hinweis: Der Anhang zählt nicht zur Seitenangabe der Arbeit
%%%%%%%%%%%%%%%%%%%%%%%%%%%%%%%%%%%%%%%%%%%%%%%%%%%%%%%%%%%%%%%%%%%%%%%%%%%%%%%%

\chapter{Anhang}
\label{chap:anhang}

%% TODO: Fügen Sie hier Ihre ergänzenden Materialien ein.
%% Der Anhang sollte nur Material enthalten, das für das Verständnis
%% der Arbeit relevant, aber zu umfangreich für den Hauptteil ist.

\section{Interviewleitfäden}
\label{sec:interview-leitfaden}
\label{app:interviewleitfaden}

%% Beispiel für einen Interviewleitfaden

\subsection{Leitfaden [Interviewtyp 1]}

\textbf{Interviewziel:} [Beschreibung des Ziels dieses Interviews]

\textbf{Zielgruppe:} [Beschreibung der Interviewpartner]

\textbf{Dauer:} ca. [X] Minuten

\textbf{Einstiegsfragen:}
\begin{enumerate}
    \item {[Einstiegsfrage zur Person/Rolle]}
    \item {[Einstiegsfrage zum Kontext]}
\end{enumerate}

\textbf{Hauptfragen:}
\begin{enumerate}
    \item {[Frage zu Themenbereich 1]}
    \item {[Frage zu Themenbereich 2]}
    \item {[Frage zu Themenbereich 3]}
    \item {[Frage zu Herausforderungen]}
    \item {[Frage zu Verbesserungspotenzialen]}
\end{enumerate}

\textbf{Abschlussfragen:}
\begin{enumerate}
    \item Gibt es noch etwas, das Sie hinzufügen möchten?
    \item Haben Sie Fragen an mich?
\end{enumerate}

\subsection{Leitfaden [Interviewtyp 2]}

%% Weiterer Interviewleitfaden nach Bedarf
[Analog zum obigen Muster]

\section{Fragebögen}
\label{sec:fragebogen}

%% Beispiel für einen standardisierten Fragebogen

\subsection{Fragebogen [Thema/Zielgruppe]}

\textbf{Einleitung:} [Text der Einleitung für Befragte]

\textbf{Teil A: Demografische Angaben}
\begin{enumerate}
    \item Ihre Position/Funktion: \_\_\_\_\_\_\_\_\_\_\_\_
    \item Berufserfahrung in Jahren: \_\_\_\_
    \item {[Weitere demografische Fragen]}
\end{enumerate}

\textbf{Teil B: [Themenbereich]}

Bitte bewerten Sie die folgenden Aussagen auf einer Skala von 1 (stimme überhaupt nicht zu) bis 5 (stimme voll zu):

\begin{enumerate}
    \item {[Aussage 1]} \hfill {[1]} {[2]} {[3]} {[4]} {[5]}
    \item {[Aussage 2]} \hfill {[1]} {[2]} {[3]} {[4]} {[5]}
    \item {[Aussage 3]} \hfill {[1]} {[2]} {[3]} {[4]} {[5]}
    \item {[Aussage 4]} \hfill {[1]} {[2]} {[3]} {[4]} {[5]}
\end{enumerate}

\textbf{Teil C: Offene Fragen}
\begin{enumerate}
    \item {[Offene Frage 1]}
    \item {[Offene Frage 2]}
\end{enumerate}

\section{Prozessdiagramme}
\label{sec:prozessdiagramme}

%% Platzhalter für Prozessdiagramme (BPMN, Flowcharts, etc.)

\subsection{[Prozess 1]}

%% Beispiel für Einbindung eines Prozessdiagramms:
%% \begin{figure}[htbp]
%% \centering
%% \includegraphics[width=\textwidth]{anhang/prozess1.pdf}
%% \caption{[Beschreibung des Prozessdiagramms]}
%% \label{fig:prozess1}
%% \end{figure}

[Hier würde das Prozessdiagramm für [Prozess 1] eingefügt]

\subsection{[Prozess 2]}

[Hier würde das Prozessdiagramm für [Prozess 2] eingefügt]

\section{Detaillierte Datenauswertungen}
\label{sec:daten-detail}

%% Detaillierte Tabellen und Statistiken, die zu umfangreich für den Hauptteil sind

\subsection{[Datenbereich 1]}

\begin{table}[htbp]
\centering
\small
\begin{tabular}{@{}lcccc@{}}
\toprule
\textbf{[Kategorie]} & \textbf{[Metrik 1]} & \textbf{[Metrik 2]} & \textbf{[Metrik 3]} & \textbf{[Metrik 4]} \\
\midrule
{[Zeile 1]} & {[X]} & {[Y]} & {[Z]} & {[W]} \\
{[Zeile 2]} & {[X]} & {[Y]} & {[Z]} & {[W]} \\
{[Zeile 3]} & {[X]} & {[Y]} & {[Z]} & {[W]} \\
\bottomrule
\end{tabular}
\caption{[Beschreibung der Detailtabelle]}
\label{tab:detail-daten}
\end{table}

\section{Technische Dokumentation}
\label{sec:tech-doku}

%% Falls technische Details dokumentiert werden müssen

\subsection{Systemkonfiguration}

\begin{table}[htbp]
\centering
\small
\begin{tabular}{@{}lll@{}}
\toprule
\textbf{Komponente} & \textbf{Version/Typ} & \textbf{Beschreibung} \\
\midrule
{[Komponente 1]} & {[Version]} & {[Kurzbeschreibung]} \\
{[Komponente 2]} & {[Version]} & {[Kurzbeschreibung]} \\
{[Komponente 3]} & {[Version]} & {[Kurzbeschreibung]} \\
\bottomrule
\end{tabular}
\caption{[Beschreibung der technischen Konfiguration]}
\label{tab:tech-config}
\end{table}

\subsection{Benutzerrollen und Berechtigungen}

\begin{table}[htbp]
\centering
\small
\begin{tabular}{@{}lp{8cm}@{}}
\toprule
\textbf{Rolle} & \textbf{Berechtigungen} \\
\midrule
{[Rolle 1]} & {[Beschreibung der Berechtigungen]} \\
{[Rolle 2]} & {[Beschreibung der Berechtigungen]} \\
{[Rolle 3]} & {[Beschreibung der Berechtigungen]} \\
\bottomrule
\end{tabular}
\caption{[Beschreibung der Rollentabelle]}
\label{tab:rollen}
\end{table}

\section{Screenshots und Bildschirmfotos}
\label{sec:screenshots}

%% Platzhalter für Screenshots

\subsection{[System/Bereich 1]}

%% Beispiel für Einbindung eines Screenshots:
%% \begin{figure}[htbp]
%% \centering
%% \includegraphics[width=0.8\textwidth]{anhang/screenshot1.png}
%% \caption{[Beschreibung des Screenshots]}
%% \label{fig:screenshot1}
%% \end{figure}

[Hier würde der Screenshot eingefügt]

\section{Glossar}
\label{sec:glossar}

%% Alphabetisch sortiertes Glossar wichtiger Fachbegriffe

\begin{description}
    \item[[Begriff 1]] [Definition des Begriffs]
    \item[[Begriff 2]] [Definition des Begriffs]
    \item[[Begriff 3]] [Definition des Begriffs]
    \item[[Begriff 4]] [Definition des Begriffs]
    \item[[Begriff 5]] [Definition des Begriffs]
\end{description}

\section{Abkürzungsverzeichnis}
\label{sec:abkuerzungen}

%% Liste der verwendeten Abkürzungen

\begin{tabular}{@{}ll@{}}
{[Abk. 1]} & {[Ausgeschriebene Form]} \\
{[Abk. 2]} & {[Ausgeschriebene Form]} \\
{[Abk. 3]} & {[Ausgeschriebene Form]} \\
{[Abk. 4]} & {[Ausgeschriebene Form]} \\
{[Abk. 5]} & {[Ausgeschriebene Form]} \\
\end{tabular}

