% !TeX root = main-limak-thesis.tex

%%%%%%%%%%%%%%%%%%%%%%%%%%%%%%%%%%%%%%%%%%%%%%%%%%%%%%%%%%%%%%%%%%%%%%%%%%%%%%%%
%% KAPITEL 3: METHODIK
%%
%% Gemäß LIMAK Leitfaden: "Methodische Vorgangsweise"
%%
%% Tipps:
%% - Beschreiben Sie Ihr Forschungsdesign und begründen Sie die Wahl
%% - Qualität vor Quantität: Die Methode(n) sollte(n) einwandfrei angewendet werden
%% - Erklären Sie, wie Sie Daten erheben und auswerten
%% - Diskutieren Sie Limitationen Ihrer Methodik
%%%%%%%%%%%%%%%%%%%%%%%%%%%%%%%%%%%%%%%%%%%%%%%%%%%%%%%%%%%%%%%%%%%%%%%%%%%%%%%%

\chapter{Methodik}
\label{chap:methodik}

%% TODO: Passen Sie die Methodik an Ihr Forschungsdesign an

\section{Forschungsdesign}
\label{sec:forschungsdesign}

%% Beschreiben Sie Ihren gewählten Forschungsansatz und begründen Sie ihn

Diese Arbeit folgt einem [qualitativ/quantitativ/mixed-methods] Forschungsansatz. Das Forschungsdesign umfasst [Beschreibung des Designs, z.B. Fallstudie, Experiment, Umfrage]. Die Wahl dieses Designs begründet sich durch [Begründung].

\subsection{Forschungsansatz}

%% Wählen Sie den passenden Abschnitt für Ihren Ansatz

Für diese Arbeit wurde ein [induktiver/deduktiver/abduktiver] Forschungsansatz gewählt. Dies bedeutet, dass [Erklärung des Ansatzes und seiner Implikationen für Ihre Arbeit].

\subsection{Fallstudiendesign}
%% Falls Sie eine Fallstudie durchführen

[Unternehmen/Organisation] dient als Fallstudie für [Forschungsthema]. Die Wahl dieser Fallstudie begründet sich durch:
\begin{itemize}
    \item {[Auswahlkriterium 1]}
    \item {[Auswahlkriterium 2]}
    \item {[Repräsentativität/Besonderheit]}
\end{itemize}

\section{Vorgehen und Phasen}
\label{sec:phasen}

%% Beschreiben Sie den Ablauf Ihrer Forschung

Die Durchführung erfolgt in [Anzahl] Hauptphasen:

\begin{enumerate}
    \item \textbf{Phase 1 -- [Name]:} [Beschreibung und Ziel der Phase]
    \item \textbf{Phase 2 -- [Name]:} [Beschreibung und Ziel der Phase]
    \item \textbf{Phase 3 -- [Name]:} [Beschreibung und Ziel der Phase]
    \item \textbf{Phase 4 -- [Name]:} [Beschreibung und Ziel der Phase]
\end{enumerate}

\section{Datenerhebung}
\label{sec:datenerhebung}

\subsection{Qualitative Methoden}
%% Falls Sie qualitative Methoden verwenden

\subsubsection{Interviews}

Es wurden [Anzahl] [strukturierte/semi-strukturierte/unstrukturierte] Interviews mit [Zielgruppe] durchgeführt. Die Auswahl der Interview\-partner*innen erfolgte nach folgenden Kriterien:
\begin{itemize}
    \item {[Kriterium 1]}
    \item {[Kriterium 2]}
    \item {[Kriterium 3]}
\end{itemize}

Der Interviewleitfaden (siehe Anhang~\ref{app:interviewleitfaden}) umfasst Fragen zu:
\begin{itemize}
    \item {[Themenbereich 1]}
    \item {[Themenbereich 2]}
    \item {[Themenbereich 3]}
\end{itemize}

\subsubsection{Dokumentenanalyse}

Folgende Dokumente wurden analysiert:
\begin{itemize}
    \item {[Dokumenttyp 1, z.B. interne Berichte]}
    \item {[Dokumenttyp 2, z.B. Prozessdokumentationen]}
    \item {[Dokumenttyp 3, z.B. Strategiepapiere]}
\end{itemize}

\subsection{Quantitative Methoden}
%% Falls Sie quantitative Methoden verwenden

\subsubsection{Fragebogendesign}

Der Fragebogen wurde auf Basis von [theoretische Grundlage/validiertes Instrument] entwickelt. Er umfasst [Anzahl] Items zu folgenden Konstrukten:
\begin{itemize}
    \item {[Konstrukt 1] ([Anzahl] Items, [Skala])}
    \item {[Konstrukt 2] ([Anzahl] Items, [Skala])}
    \item {[Konstrukt 3] ([Anzahl] Items, [Skala])}
\end{itemize}

\subsubsection{Stichprobe}

Die Stichprobe umfasst [Anzahl] Teilnehmende. Die Auswahl erfolgte durch [Auswahlverfahren]. Die Rücklaufquote betrug [X]\%.

\subsection{Kennzahlen und Metriken}

%% Definieren Sie die Kennzahlen, die Sie erheben

Folgende Kennzahlen werden erhoben:

\begin{table}[htbp]
\centering
\begin{tabular}{@{}lll@{}}
\toprule
\textbf{Kennzahl} & \textbf{Definition} & \textbf{Messmethode} \\
\midrule
{[Kennzahl 1]} & {[Definition]} & {[Wie gemessen]} \\
{[Kennzahl 2]} & {[Definition]} & {[Wie gemessen]} \\
{[Kennzahl 3]} & {[Definition]} & {[Wie gemessen]} \\
{[Kennzahl 4]} & {[Definition]} & {[Wie gemessen]} \\
\bottomrule
\end{tabular}
\caption{Übersicht der erhobenen Kennzahlen}
\label{tab:kennzahlen}
\end{table}

\section{Datenauswertung}
\label{sec:auswertung}

\subsection{Qualitative Auswertung}

%% Falls qualitative Daten ausgewertet werden

Die qualitative Auswertung erfolgt mittels [Methode, z.B. qualitative Inhaltsanalyse nach Mayring, Grounded Theory]. Das Vorgehen umfasst:
\begin{enumerate}
    \item {[Schritt 1, z.B. Transkription]}
    \item {[Schritt 2, z.B. Kodierung]}
    \item {[Schritt 3, z.B. Kategorienbildung]}
    \item {[Schritt 4, z.B. Interpretation]}
\end{enumerate}

\subsection{Quantitative Auswertung}

%% Falls quantitative Daten ausgewertet werden

Die quantitative Auswertung erfolgt mittels [Software, z.B. SPSS, R, Excel]. Folgende statistische Verfahren werden eingesetzt:
\begin{itemize}
    \item Deskriptive Statistik (Mittelwerte, Standardabweichungen)
    \item {[Weitere Verfahren, z.B. Korrelationsanalyse, Regression]}
    \item {[Testverfahren, z.B. t-Test, ANOVA]}
\end{itemize}

\section{Gütekriterien}
\label{sec:guetekriterien}

%% Diskutieren Sie die Qualität Ihrer Forschung

\subsection{Validität}

Die Validität der Ergebnisse wird durch folgende Maßnahmen sichergestellt:
\begin{itemize}
    \item {[Maßnahme 1, z.B. Triangulation]}
    \item {[Maßnahme 2, z.B. Member Checking]}
    \item {[Maßnahme 3, z.B. Peer Debriefing]}
\end{itemize}

\subsection{Reliabilität}

Die Reliabilität wird gewährleistet durch:
\begin{itemize}
    \item {[Maßnahme 1]}
    \item {[Maßnahme 2]}
\end{itemize}

\section{Ethische Aspekte und Datenschutz}
\label{sec:ethik}

Die Teilnahme an [Interviews/Befragungen] erfolgte freiwillig. Alle Teilnehmenden wurden über [Zweck der Studie/Verwendung der Daten] informiert. Die Daten werden anonymisiert ausgewertet und [weitere Datenschutzmaßnahmen].

\section{Limitationen der Methodik}
\label{sec:limitationen-methodik}

%% Diskutieren Sie ehrlich die Grenzen Ihrer Methodik

Folgende Limitationen sind bei der Interpretation der Ergebnisse zu berücksichtigen:
\begin{itemize}
    \item \textbf{[Limitation 1]:} [Beschreibung und Auswirkung]
    \item \textbf{[Limitation 2]:} [Beschreibung und Auswirkung]
    \item \textbf{[Limitation 3]:} [Beschreibung und Auswirkung]
\end{itemize}
