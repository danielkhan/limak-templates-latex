% !TeX root = main-limak-thesis.tex

%%%%%%%%%%%%%%%%%%%%%%%%%%%%%%%%%%%%%%%%%%%%%%%%%%%%%%%%%%%%%%%%%%%%%%%%%%%%%%%%
%% KAPITEL 9: ZUSAMMENFASSUNG UND AUSBLICK
%%
%% Gemäß LIMAK Leitfaden: Conclusio mit Implikationen, Limitationen und Ausblick
%%
%% Tipps:
%% - Kompakte Zusammenfassung der gesamten Arbeit
%% - Keine neuen Informationen einführen
%% - Klare Beantwortung der Forschungsfrage
%% - Praktische und wissenschaftliche Implikationen
%% - Ausblick auf zukünftige Forschung
%% - Abschließende Schlussbemerkung
%%%%%%%%%%%%%%%%%%%%%%%%%%%%%%%%%%%%%%%%%%%%%%%%%%%%%%%%%%%%%%%%%%%%%%%%%%%%%%%%

\chapter{Zusammenfassung und Ausblick}
\label{chap:zusammenfassung}

%% TODO: Ersetzen Sie alle Platzhalter [...] mit Ihren eigenen Inhalten.
%% Fassen Sie Ihre gesamte Arbeit prägnant zusammen.

\section{Zusammenfassung der Ergebnisse}
\label{sec:zusammenfassung-ergebnisse}

Diese Masterarbeit untersuchte [Thema/Forschungsgegenstand]. Die zentrale Forschungsfrage fokussierte auf [Kern der Forschungsfrage].

\subsection{Methodisches Vorgehen}

Die Arbeit folgte einem strukturierten Vorgehen:
\begin{enumerate}
    \item \textbf{[Phase 1]:} [Kurze Beschreibung]
    \item \textbf{[Phase 2]:} [Kurze Beschreibung]
    \item \textbf{[Phase 3]:} [Kurze Beschreibung]
    \item \textbf{[Phase 4]:} [Kurze Beschreibung]
\end{enumerate}

\subsection{Zentrale Erkenntnisse}

Die Untersuchung lieferte folgende zentrale Erkenntnisse:

\textbf{Zu [Themenbereich 1]:}
\begin{itemize}
    \item {[Zentrale Erkenntnis 1]}
    \item {[Zentrale Erkenntnis 2]}
    \item {[Zentrale Erkenntnis 3]}
\end{itemize}

\textbf{Zu [Themenbereich 2]:}
\begin{itemize}
    \item {[Zentrale Erkenntnis 1]}
    \item {[Zentrale Erkenntnis 2]}
    \item {[Zentrale Erkenntnis 3]}
\end{itemize}

\textbf{Zu [Themenbereich 3]:}
\begin{itemize}
    \item {[Zentrale Erkenntnis 1]}
    \item {[Zentrale Erkenntnis 2]}
\end{itemize}

\subsection{Beantwortung der Forschungsfrage}

Die Forschungsfrage kann wie folgt beantwortet werden: [Prägnante, zusammenfassende Beantwortung der Forschungsfrage in 2-3 Sätzen]

[Ergänzende Erläuterung der wichtigsten Erkenntnisse, die zur Beantwortung beitragen]

\section{Praktische Implikationen}
\label{sec:implikationen}

\subsection{Für [Zielgruppe 1 -- z.B. das untersuchte Unternehmen]}

Die Ergebnisse haben folgende praktische Bedeutung:
\begin{itemize}
    \item {[Implikation 1]}
    \item {[Implikation 2]}
    \item {[Implikation 3]}
\end{itemize}

\subsection{Für [Zielgruppe 2 -- z.B. andere Unternehmen/Manager]}

Für [Zielgruppe] lassen sich folgende Handlungsempfehlungen ableiten:
\begin{itemize}
    \item {[Handlungsempfehlung 1]}
    \item {[Handlungsempfehlung 2]}
    \item {[Handlungsempfehlung 3]}
\end{itemize}

\section{Ausblick}
\label{sec:ausblick}

\subsection{Kurzfristige Entwicklungen}

%% Was sollte/könnte unmittelbar umgesetzt werden?
\begin{itemize}
    \item {[Kurzfristige Maßnahme/Entwicklung 1]}
    \item {[Kurzfristige Maßnahme/Entwicklung 2]}
    \item {[Kurzfristige Maßnahme/Entwicklung 3]}
\end{itemize}

\subsection{Mittelfristige Entwicklungen}

%% Was könnte in 1-2 Jahren relevant werden?
\begin{itemize}
    \item {[Mittelfristige Entwicklung 1]}
    \item {[Mittelfristige Entwicklung 2]}
    \item {[Mittelfristige Entwicklung 3]}
\end{itemize}

\subsection{Forschungsperspektiven}

Aufbauend auf dieser Arbeit könnten folgende Forschungsfragen untersucht werden:
\begin{itemize}
    \item {[Forschungsfrage 1 -- z.B. Langzeitstudie]}
    \item {[Forschungsfrage 2 -- z.B. Vergleichsstudie]}
    \item {[Forschungsfrage 3 -- z.B. Erweiterung auf andere Kontexte]}
    \item {[Forschungsfrage 4 -- z.B. vertiefende Analyse einzelner Aspekte]}
\end{itemize}

\section{Schlussbemerkung}
\label{sec:schlussbemerkung}

%% Abschließende Reflexion und persönliches Fazit (2-3 Absätze)

[Abschließende Reflexion zur Bedeutung der Arbeit und der gewonnenen Erkenntnisse]

[Zusammenfassende Aussage zum Beitrag der Arbeit für Theorie und/oder Praxis]

[Optionaler Ausblick oder abschließender Gedanke]

