% !TeX root = main-limak-thesis.tex

%%%%%%%%%%%%%%%%%%%%%%%%%%%%%%%%%%%%%%%%%%%%%%%%%%%%%%%%%%%%%%%%%%%%%%%%%%%%%%%%
%% KAPITEL 8: DISKUSSION
%%
%% Gemäß LIMAK Leitfaden: Interpretation und kritische Reflexion der Ergebnisse
%%
%% Tipps:
%% - Interpretation der Ergebnisse im Kontext der Forschungsfrage
%% - Einordnung in den theoretischen Rahmen
%% - Vergleich mit anderen Studien/Ansätzen
%% - Kritische Reflexion der eigenen Arbeit
%% - Limitationen transparent benennen
%% - Unterschied zu Kapitel 7: Hier interpretieren Sie, dort beschreiben Sie nur
%%%%%%%%%%%%%%%%%%%%%%%%%%%%%%%%%%%%%%%%%%%%%%%%%%%%%%%%%%%%%%%%%%%%%%%%%%%%%%%%

\chapter{Diskussion}
\label{chap:diskussion}

%% TODO: Passen Sie die Abschnitte an Ihre Forschungsfrage an und ersetzen Sie
%% alle Platzhalter [...] mit Ihren eigenen Interpretationen und Reflexionen.

%% Kurze Einleitung zum Kapitel
In diesem Kapitel werden die in Kapitel~\ref{chap:evaluation} präsentierten Ergebnisse interpretiert und kritisch reflektiert. Zunächst erfolgt die Beantwortung der Forschungsfrage (Abschnitt~\ref{sec:interpretation}), gefolgt von einer Einordnung in den theoretischen Kontext (Abschnitt~\ref{sec:theoretischer-kontext}). Abschließend werden Limitationen und die Übertragbarkeit der Ergebnisse diskutiert.

\section{Interpretation der Ergebnisse}
\label{sec:interpretation}

\subsection{Beantwortung der Forschungsfrage}

Die zentrale Forschungsfrage dieser Arbeit lautete:

\begin{quote}
\textit{[Ihre Forschungsfrage aus Kapitel 1 hier einfügen]}
\end{quote}

Basierend auf den erhobenen Daten lässt sich die Forschungsfrage wie folgt beantworten:

[Systematische Beantwortung der Forschungsfrage unter Bezugnahme auf die konkreten Ergebnisse aus Kapitel 7. Strukturieren Sie nach den Teilaspekten Ihrer Forschungsfrage.]

\textbf{Zu Teilaspekt 1:} [Interpretation]

\textbf{Zu Teilaspekt 2:} [Interpretation]

\textbf{Zu Teilaspekt 3:} [Interpretation]

\subsection{Interpretation der Hauptergebnisse}

[Detaillierte Interpretation der wichtigsten Ergebnisse. Was bedeuten die Zahlen/Befunde konkret? Warum sind bestimmte Ergebnisse so ausgefallen?]

\section{Einordnung in den theoretischen Kontext}
\label{sec:theoretischer-kontext}

\subsection{Bezug zu [Theoriebereich 1]}

Die Ergebnisse dieser Arbeit bestätigen/widersprechen/ergänzen die theoretischen Erkenntnisse von \textcite{Autor2020} insofern, als [Erläuterung].

%% Systematischer Abgleich mit dem in Kapitel 2 dargestellten theoretischen Rahmen

\subsection{Bezug zu [Theoriebereich 2]}

Im Hinblick auf [Konzept/Modell] zeigen die Ergebnisse, dass [Interpretation im Kontext der Theorie].

\subsection{Implikationen für bestehende Theorien}

[Welche Implikationen haben Ihre Ergebnisse für die bestehenden theoretischen Modelle/Konzepte?]

\section{Vergleich mit bestehenden Ansätzen}
\label{sec:vergleich}

Im Vergleich zu [bisherigen Studien/Ansätzen] zeigt diese Arbeit folgende Gemeinsamkeiten und Unterschiede:

\begin{itemize}
    \item \textbf{Gemeinsamkeit 1:} [Beschreibung]
    \item \textbf{Unterschied 1:} [Beschreibung und mögliche Erklärung]
    \item \textbf{Unterschied 2:} [Beschreibung und mögliche Erklärung]
\end{itemize}

\section{Kritische Reflexion}
\label{sec:reflexion}

\subsection{Limitationen der Studie}

Folgende Limitationen sind bei der Interpretation der Ergebnisse zu berücksichtigen:

\textbf{Methodische Limitationen:}
\begin{itemize}
    \item {[Limitation 1 -- z.B. Stichprobengröße, Auswahlverfahren]}
    \item {[Limitation 2 -- z.B. Erhebungsinstrumente]}
    \item {[Limitation 3 -- z.B. Zeitrahmen der Untersuchung]}
\end{itemize}

\textbf{Inhaltliche Limitationen:}
\begin{itemize}
    \item {[Limitation 1 -- z.B. Fokus auf bestimmte Aspekte]}
    \item {[Limitation 2 -- z.B. nicht berücksichtigte Einflussfaktoren]}
\end{itemize}

\textbf{Kontextbezogene Limitationen:}
\begin{itemize}
    \item {[Limitation 1 -- z.B. Branchenspezifika]}
    \item {[Limitation 2 -- z.B. regionale Besonderheiten]}
\end{itemize}

\subsection{Stärken der Studie}

Trotz der genannten Limitationen weist die Studie folgende Stärken auf:
\begin{itemize}
    \item {[Stärke 1]}
    \item {[Stärke 2]}
    \item {[Stärke 3]}
\end{itemize}

\section{Übertragbarkeit der Ergebnisse}
\label{sec:uebertragbarkeit}

\subsection{Generalisierbarkeit}

Die Ergebnisse dieser Einzelfallstudie/empirischen Untersuchung sind unter folgenden Bedingungen auf andere Kontexte übertragbar:
\begin{itemize}
    \item {[Bedingung 1 -- z.B. ähnliche Unternehmensgröße]}
    \item {[Bedingung 2 -- z.B. vergleichbare Branche]}
    \item {[Bedingung 3 -- z.B. ähnliche Ausgangssituation]}
\end{itemize}

\subsection{Einschränkungen der Übertragbarkeit}

Folgende Faktoren schränken die Übertragbarkeit ein:
\begin{itemize}
    \item {[Einschränkung 1]}
    \item {[Einschränkung 2]}
\end{itemize}

\section{Beitrag zur Praxis}
\label{sec:beitrag-praxis}

Diese Arbeit liefert folgende praktische Beiträge für [Zielgruppe]:
\begin{itemize}
    \item {[Praktischer Beitrag 1 -- z.B. Handlungsempfehlungen]}
    \item {[Praktischer Beitrag 2 -- z.B. Vorgehensmodell]}
    \item {[Praktischer Beitrag 3 -- z.B. Best Practices]}
    \item {[Praktischer Beitrag 4 -- z.B. Checklisten, Tools]}
\end{itemize}

\section{Beitrag zur Wissenschaft}
\label{sec:beitrag-wissenschaft}

Wissenschaftlich trägt diese Arbeit bei durch:
\begin{itemize}
    \item {[Wissenschaftlicher Beitrag 1 -- z.B. empirische Daten]}
    \item {[Wissenschaftlicher Beitrag 2 -- z.B. Theorieweiterentwicklung]}
    \item {[Wissenschaftlicher Beitrag 3 -- z.B. methodische Erkenntnisse]}
\end{itemize}

