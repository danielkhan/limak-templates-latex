% !TeX root = main-limak-thesis.tex

%%%%%%%%%%%%%%%%%%%%%%%%%%%%%%%%%%%%%%%%%%%%%%%%%%%%%%%%%%%%%%%%%%%%%%%%%%%%%%%%
%% KAPITEL 6: IMPLEMENTIERUNG / UMSETZUNG
%%
%% Gemäß LIMAK Leitfaden: "Ggf. Implementierungsplan"
%%
%% Dieses Kapitel beschreibt die praktische Umsetzung des Konzepts.
%% Je nach Thema kann es sich um technische oder organisatorische
%% Implementierung handeln.
%%%%%%%%%%%%%%%%%%%%%%%%%%%%%%%%%%%%%%%%%%%%%%%%%%%%%%%%%%%%%%%%%%%%%%%%%%%%%%%%

\chapter{Implementierung}
\label{chap:implementierung}

%% TODO: Passen Sie den Inhalt an Ihr Projekt an

\section{Implementierungsstrategie}
\label{sec:impl-strategie}

\subsection{Vorgehensweise}

Die Implementierung erfolgt nach folgendem Ansatz:
\begin{itemize}
    \item {[Beschreibung des gewählten Ansatzes, z.B. phasenweise, agil, Big-Bang]}
    \item {[Begründung für die Wahl]}
\end{itemize}

\subsection{Rahmenbedingungen}

Folgende Rahmenbedingungen sind zu berücksichtigen:
\begin{itemize}
    \item \textbf{Zeitrahmen:} [Zeitraum]
    \item \textbf{Budget:} [falls relevant]
    \item \textbf{Ressourcen:} [beteiligte Personen/Teams]
    \item \textbf{Abhängigkeiten:} [externe Faktoren]
\end{itemize}

\section{Implementierungsphasen}
\label{sec:impl-phasen}

\subsection{Phase 1: [Name der Phase]}

\textbf{Ziel:} [Was soll erreicht werden?]

\textbf{Aktivitäten:}
\begin{enumerate}
    \item {[Aktivität 1]}
    \item {[Aktivität 2]}
    \item {[Aktivität 3]}
\end{enumerate}

\textbf{Ergebnisse:} [Erwartete Deliverables]

\subsection{Phase 2: [Name der Phase]}

[Analog zu Phase 1 strukturieren]

\subsection{Phase 3: [Name der Phase]}

[Analog zu Phase 1 strukturieren]

\section{Detaillierte Umsetzung}
\label{sec:umsetzung-detail}

%% Beschreiben Sie die konkrete Umsetzung in den einzelnen Bereichen

\subsection{[Umsetzungsbereich 1]}

[Detaillierte Beschreibung der Umsetzung, z.B. Konfiguration, Prozessanpassung, Systemeinführung]

\textbf{Durchgeführte Maßnahmen:}
\begin{itemize}
    \item {[Maßnahme 1]}
    \item {[Maßnahme 2]}
    \item {[Maßnahme 3]}
\end{itemize}

\subsection{[Umsetzungsbereich 2]}

[Analog strukturieren]

\section{Change Management}
\label{sec:change-management}

%% Beschreiben Sie wie die Veränderung begleitet wurde

\subsection{Kommunikation}

[Wie wurde über die Veränderung kommuniziert?]

\subsection{Schulung und Qualifizierung}

\begin{itemize}
    \item {[Schulungsmaßnahme 1]}
    \item {[Schulungsmaßnahme 2]}
    \item {[Schulungsmaßnahme 3]}
\end{itemize}

\subsection{Begleitung der Einführung}

[Wie wurde der Go-Live begleitet? Support, Pilotphasen, etc.]

\section{Herausforderungen und Lösungen}
\label{sec:herausforderungen}

%% Dokumentieren Sie aufgetretene Probleme und deren Lösung

\begin{table}[htbp]
\centering
\begin{tabular}{@{}p{4cm}p{5cm}p{4cm}@{}}
\toprule
\textbf{Herausforderung} & \textbf{Beschreibung} & \textbf{Lösungsansatz} \\
\midrule
{[Herausforderung 1]} & {[Beschreibung]} & {[Wie gelöst?]} \\
{[Herausforderung 2]} & {[Beschreibung]} & {[Wie gelöst?]} \\
{[Herausforderung 3]} & {[Beschreibung]} & {[Wie gelöst?]} \\
\bottomrule
\end{tabular}
\caption{Herausforderungen und Lösungsansätze während der Implementierung}
\label{tab:herausforderungen}
\end{table}

\section{Status der Implementierung}
\label{sec:impl-status}

%% Dokumentieren Sie den aktuellen Stand

Zum Zeitpunkt der Fertigstellung dieser Arbeit ist die Implementierung zu [X]\% abgeschlossen. Folgende Komponenten sind bereits umgesetzt:
\begin{itemize}
    \item {[Komponente 1] -- Status: [fertig/in Arbeit/geplant]}
    \item {[Komponente 2] -- Status: [fertig/in Arbeit/geplant]}
    \item {[Komponente 3] -- Status: [fertig/in Arbeit/geplant]}
\end{itemize}
