% !TeX program = xelatex
% !TeX encoding = UTF-8
% !TeX spellcheck = de_DE
% !BIB program = biber
%%
%% The above lines help editors like TeXstudio to automatically choose the right tools
%% to compile your LaTeX source file. If your tool does not support these magic comments,
%% you will need to make appropriate manual choices.
%%
%% You can safely use "pdflatex" instead of "xelatex" if you prefer the pdfLaTeX toolchain.
%% However, pdfLaTeX will not be able to deliver the professional font experience that you
%% will get with XeLaTeX. You can also safely use "lualatex" instead of "xelatex" while
%% preserving the professional font experience if you prefer the LuaLaTeX toolchain.
%%
%% _Important_: These magic comments should be on the first lines of your source file. 
%%
%%%%%%%%%%%%%%%%%%%%%%%%%%%%%%%%%%%%%%%%%%%%%%%%%%%%%%%%%%%%%%%%%%%%%%%%%%%%%%%%

%%%%%%%%%%%%%%%%%%%%%%%%%%%%%%%%%%%%%%%%%%%%%%%%%%%%%%%%%%%%%%%%%%%%%%%%%%%%%%%%
%%
%% LIMAK LaTeX Masterarbeits-Vorlage
%%
%% Basierend auf der JKU LaTeX Technical Report Vorlage von Michael Roland (2021)
%% Angepasst für LIMAK Johannes Kepler Universität - Master in Management
%%
%%%%%%%%%%%%%%%%%%%%%%%%%%%%%%%%%%%%%%%%%%%%%%%%%%%%%%%%%%%%%%%%%%%%%%%%%%%%%%%%

%%%%%%%%%%%%%%%%%%%%%%%%%%%%%%%%%%%%%%%%%%%%%%%%%%%%%%%%%%%%%%%%%%%%%%%%%%%%%%%%
%%
%% Document class: This is a koma-script book.
%%
\documentclass[a4paper,oneside,11pt,english,ngerman,listof=totoc]{scrbook}
%%
%% The comma-separated list in square brackets are class options.
%% Useful options that you might want to use: 
%%
%% Paper size:
%%  * a4paper ... A4 paper size
%%
%% Optimize for single-sided or double-sided printing:
%%  * oneside ... single-sided 
%%  * twoside ... double-sided
%%
%% Base font size:
%%  * 10pt ... 10-pt font is used for normal text
%%  * 11pt ... 11-pt font is used for normal text (LIMAK Standard = Arial 11pt)
%%  * 12pt ... 12-pt font is used for normal text 
%%
%% Define document languages (the last specified language becomes the document default
%% language):
%%  * ngerman ... German (Hauptsprache für LIMAK-Arbeiten)
%%  * english ... English
%%
%%%%%%%%%%%%%%%%%%%%%%%%%%%%%%%%%%%%%%%%%%%%%%%%%%%%%%%%%%%%%%%%%%%%%%%%%%%%%%%%

%%%%%%%%%%%%%%%%%%%%%%%%%%%%%%%%%%%%%%%%%%%%%%%%%%%%%%%%%%%%%%%%%%%%%%%%%%%%%%%%
%%
%% Treat input files as UTF-8 encoded. Make sure to always load this when you use pdfLaTeX
%% so that pdfLaTeX knows how to read and interpret characters in this source file.
%%
\usepackage[utf8]{inputenc}
%%
%%%%%%%%%%%%%%%%%%%%%%%%%%%%%%%%%%%%%%%%%%%%%%%%%%%%%%%%%%%%%%%%%%%%%%%%%%%%%%%%

%%%%%%%%%%%%%%%%%%%%%%%%%%%%%%%%%%%%%%%%%%%%%%%%%%%%%%%%%%%%%%%%%%%%%%%%%%%%%%%%
%%
%% Use the JKU LaTeX technical report template for this document.
%%
\usepackage[mathesis,nofancyfonts,LIMAK]{jkureport}
%%
%% The comma-separated list in square brackets are theme options. Useful options that you
%% might want to use:
%%
%% Document type:
%%  * mathesis      ... Master's thesis (für LIMAK Master in Management)
%%
%% Color scheme selection options:
%%  * LIMAK ... Use LIMAK Austrian Business School color scheme (mit LIMAK-Logo)
%%  * BUS   ... Use Business School color scheme (generisch)
%%  * JKU   ... Use JKU (gray) color scheme
%%
%% Font options (requires XeTeX/LuaTeX):
%%  * fancyfonts    ... Use custom TTF fonts with XeTeX/LuaTeX
%%  * compactmono   ... Use condensed fixed-width font everywhere.
%%  * nocompactverb ... Do not use condensed fixed-width font for verbatim and listings.
%%
%%%%%%%%%%%%%%%%%%%%%%%%%%%%%%%%%%%%%%%%%%%%%%%%%%%%%%%%%%%%%%%%%%%%%%%%%%%%%%%%

%%%%%%%%%%%%%%%%%%%%%%%%%%%%%%%%%%%%%%%%%%%%%%%%%%%%%%%%%%%%%%%%%%%%%%%%%%%%%%%%
%%
%% This is the place where you can load additional packages. If you want to load
%% a package `biblatex', you would use the command `\usepackage{biblatex}'.
%%
\usepackage{csquotes}
%% LIMAK verwendet Harvard-Zitierstil (author-year) gemäß Leitfaden
%% Beispiel: (Mustermann / Musterfrau 2020, S. 15)
\usepackage[backend=biber,style=authoryear,maxcitenames=2,maxbibnames=99,sorting=nyt,uniquename=false,uniquelist=false]{biblatex}
%% "/" zwischen zwei Autoren (gemäß Leitfaden Beispiel: Johnson / Scholes 2005)
\renewcommand*{\finalnamedelim}{\addspace/\addspace}
\renewcommand*{\multinamedelim}{,\addspace}
\setcounter{biburlnumpenalty}{100} %% reducing biburl* penalties typically improves URL placement in bibliography
\setcounter{biburllcpenalty}{100}
\setcounter{biburlucpenalty}{100}
\usepackage{todonotes}
\usepackage{import}
\usepackage{amsfonts}
\usepackage{subfigure}
\usepackage{graphicx}
\usepackage{booktabs}
\usepackage{ifthen}  % Für bedingte Anweisungen auf Titelseite

%%%%%%%%%%%%%%%%%%%%%%%%%%%%%%%%%%%%%%%%%%%%%%%%%%%%%%%%%%%%%%%%%%%%%%%%%%%%%%%%
%%
%% LIMAK-spezifische Formatierung (entsprechend Word-Vorlage)
%%
%% Schriftart: Arial (statt Merriweather)
%% Schriftgröße: 11pt (Dokumentklasse)
%% Zeilenabstand: 1.5
%% Seitenränder: 3cm oben/unten/links, 2.5cm rechts
%%
\usepackage{setspace}
\onehalfspacing  % 1.5 Zeilenabstand

% Seitenränder anpassen (Word-Vorlage: 3/3/3/2.5 cm)
\usepackage{geometry}
\geometry{
    top=3cm,
    bottom=3cm,
    left=3cm,
    right=2.5cm,
    head=16.43pt  % Fix head height warning from scrlayer-scrpage
}

% Arial als Hauptschriftart (LIMAK Standard)
% Mit nofancyfonts Option wird fontspec nicht von jkureport geladen,
% daher laden wir es hier selbst für Arial
\usepackage{fontspec}
\setmainfont{Arial}
\setsansfont{Arial}

%%
%%%%%%%%%%%%%%%%%%%%%%%%%%%%%%%%%%%%%%%%%%%%%%%%%%%%%%%%%%%%%%%%%%%%%%%%%%%%%%%%

%%%%%%%%%%%%%%%%%%%%%%%%%%%%%%%%%%%%%%%%%%%%%%%%%%%%%%%%%%%%%%%%%%%%%%%%%%%%%%%%
%%
%% Bibliography data files.
%%
\addbibresource{references.bib}
%%
%%%%%%%%%%%%%%%%%%%%%%%%%%%%%%%%%%%%%%%%%%%%%%%%%%%%%%%%%%%%%%%%%%%%%%%%%%%%%%%%

\begin{document}

%%%%%%%%%%%%%%%%%%%%%%%%%%%%%%%%%%%%%%%%%%%%%%%%%%%%%%%%%%%%%%%%%%%%%%%%%%%%%%%%
%%
%% Logo-Größe und -Position anpassen für breites LIMAK-Logo
%% Das LIMAK-Logo ist breiter als Standard-JKU-Logos (Verhältnis 3.5:1)
%% Das Logo wird rechtsbündig mit dem Textblock ausgerichtet
%% WICHTIG: Diese Anpassungen müssen NACH \begin{document} stehen!
%%
\makeatletter
\setlength{\jkureport@logo@titlepage@height}{16mm}      % Standard: 25mm, für LIMAK kleiner
\setlength{\jkureport@logo@titlepage@hoffset}{12.9mm}   % Standard-Wert beibehalten für Ausrichtung
\makeatother
%%
%%%%%%%%%%%%%%%%%%%%%%%%%%%%%%%%%%%%%%%%%%%%%%%%%%%%%%%%%%%%%%%%%%%%%%%%%%%%%%%%
%% Begin with the frontmatter (among other things, this switches to roman page numbers)
\frontmatter

%%%%%%%%%%%%%%%%%%%%%%%%%%%%%%%%%%%%%%%%%%%%%%%%%%%%%%%%%%%%%%%%%%%%%%%%%%%%%%%%
%%
%% LIMAK Titelseite (gemäß offiziellem LIMAK Layout)
%%
%% TODO: Passen Sie die folgenden Variablen an Ihre Arbeit an:
%%

%% Farbe für LIMAK-Akzente (Orange)
\definecolor{LIMAKorange}{RGB}{232,119,34}

%% Titel der Arbeit
\newcommand{\LIMAKtitel}{Titel der Arbeit}

%% Kurztitel für Kopfzeile
\titleshort{Kurztitel für Kopfzeile}

%% Name des/der Teilnehmenden
\newcommand{\LIMAKautor}{Name des*der Teilnehmenden}

%% Name der Betreuungsperson
\newcommand{\LIMAKbetreuer}{Name der Betreuungsperson}

%% Akademischer Grad (Curriculum-abhängig)
%% Curriculum ab 2023W: "Executive Master of Business Administration"
%% Curriculum ab 2025S: "Master of Business Administration"
\newcommand{\LIMAKgrad}{Executive Master of Business Administration}

%% Studiengang - wählen Sie EINEN der folgenden:
%% Option 1: Executive MBA Management & Leadership
%% Option 2: Global Executive MBA
%% Bei Unsicherheit beide mit "oder" angeben (wie im Template)
\newcommand{\LIMAKstudiengang}{Executive MBA Management \& Leadership}
\newcommand{\LIMAKstudiengangZwei}{Global Executive MBA}  % Leer lassen wenn nur ein Studiengang
\newcommand{\LIMAKzeigOder}{true}  % "true" zeigt beide mit "oder", "false" zeigt nur ersten

%% Ort und Datum der Abgabe
\newcommand{\LIMAKort}{Linz}
\newcommand{\LIMAKdatum}{Monat und Jahr der Abgabe}

%% Titelseite erstellen
\begin{titlepage}
\thispagestyle{empty}

%% Logo oben rechts
\begin{tikzpicture}[remember picture,overlay]
    \node[anchor=north east, inner sep=0] at ([xshift=-30mm,yshift=-15mm]current page.north east) {
        \includegraphics[height=12mm]{logos/jku_LIMAK_black.png}
    };
\end{tikzpicture}

\vspace*{20mm}

%% Titel
{\centering
{\fontsize{22pt}{26pt}\selectfont\bfseries\LIMAKtitel\par}

\vspace{15mm}

%% Masterarbeit
{\fontsize{20pt}{24pt}\selectfont Masterarbeit\par}

\vspace{6mm}

%% Zur Erlangung...
{\fontsize{12pt}{14pt}\selectfont zur Erlangung des akademischen Grades\par}

\vspace{3mm}

%% Akademischer Grad
{\fontsize{14pt}{17pt}\selectfont „\LIMAKgrad"\par}

\vspace{8mm}

%% Im außerordentlichen Masterstudium
{\fontsize{12pt}{14pt}\selectfont im außerordentlichen Masterstudium\par}

\vspace{3mm}

%% Studiengang(e)
{\fontsize{14pt}{17pt}\selectfont
\ifthenelse{\equal{\LIMAKzeigOder}{true}}{%
    \LIMAKstudiengang{} \underline{oder}\\[2mm]
    \LIMAKstudiengangZwei%
}{%
    \LIMAKstudiengang%
}\par}

\vspace{12mm}
}

%% Linke Seite: Eingereicht von, bei, Betreuer
\begin{flushleft}
{\fontsize{11pt}{14pt}\selectfont
\textbf{Eingereicht von:}\\
\LIMAKautor

\vspace{6mm}

\textbf{Eingereicht bei:}\\
LIMAK Austrian Business School GmbH

\vspace{6mm}

\textbf{Betreuer*in:}\\
\LIMAKbetreuer

\vspace{8mm}

\LIMAKort, \textit{(\LIMAKdatum)}
}
\end{flushleft}

\end{titlepage}

%% Titel auch für PDF-Metadaten und Kopfzeilen setzen
\title{\LIMAKtitel}
\author{\LIMAKautor}
%%
%%%%%%%%%%%%%%%%%%%%%%%%%%%%%%%%%%%%%%%%%%%%%%%%%%%%%%%%%%%%%%%%%%%%%%%%%%%%%%%%

%%%%%%%%%%%%%%%%%%%%%%%%%%%%%%%%%%%%%%%%%%%%%%%%%%%%%%%%%%%%%%%%%%%%%%%%%%%%%%%%
%%
%% FRONTMATTER - Reihenfolge gemäß LIMAK Leitfaden:
%% 1. Danksagung
%% 2. Eidesstattliche Erklärung
%% 3. Inhaltsverzeichnis
%% 4. Abbildungsverzeichnis
%% 5. Tabellenverzeichnis
%% 6. Abkürzungsverzeichnis
%% 7. Kurzfassung (Deutsch)
%% 8. Abstract (Englisch)
%%
%%%%%%%%%%%%%%%%%%%%%%%%%%%%%%%%%%%%%%%%%%%%%%%%%%%%%%%%%%%%%%%%%%%%%%%%%%%%%%%%

%% 1. Danksagung
\import{./}{00-danksagung}
\addcontentsline{toc}{chapter}{Danksagung}

%% 2. Eidesstattliche Erklärung
\cleardoubleoddpage
\chapter*{Eidesstattliche Erklärung}
\addcontentsline{toc}{chapter}{Eidesstattliche Erklärung}

Ich erkläre an Eides statt, dass ich die vorliegende Masterarbeit selbstständig und ohne fremde Hilfe verfasst, andere als die angegebenen Quellen und Hilfsmittel nicht benutzt bzw. die wörtlich oder sinngemäß entnommenen Stellen als solche kenntlich gemacht habe.

Die vorliegende Masterarbeit ist mit dem elektronisch übermittelten Textdokument identisch.

\vspace{3cm}

\noindent
\begin{tabular}{@{}p{0.4\textwidth}p{0.5\textwidth}@{}}
\hrulefill & \hrulefill \\
Ort, Datum & Unterschrift \\
\end{tabular}

%% 3. Inhaltsverzeichnis
\cleardoubleoddpage
\tableofcontents
\addcontentsline{toc}{chapter}{Inhaltsverzeichnis}

%% 4. Abbildungsverzeichnis
\cleardoubleoddpage
\listoffigures

%% 5. Tabellenverzeichnis
\cleardoubleoddpage
\listoftables

%% 6. Abkürzungsverzeichnis (optional - auskommentieren wenn nicht benötigt)
\cleardoubleoddpage
\import{./}{00-abkuerzungen}
\addcontentsline{toc}{chapter}{Abkürzungsverzeichnis}

%% 7. Kurzfassung (Deutsch) und 8. Abstract (Englisch)
\import{./}{00-abstract}
%%
%%%%%%%%%%%%%%%%%%%%%%%%%%%%%%%%%%%%%%%%%%%%%%%%%%%%%%%%%%%%%%%%%%%%%%%%%%%%%%%%

%% Begin with the mainmatter (among other things, this switches to arabic page numbers)
\mainmatter

%%%%%%%%%%%%%%%%%%%%%%%%%%%%%%%%%%%%%%%%%%%%%%%%%%%%%%%%%%%%%%%%%%%%%%%%%%%%%%%%
%%
%% Include your chapters
%%
\import{./}{01-einleitung}
\import{./}{02-theoretische-grundlagen}
\import{./}{03-methodik}
\import{./}{04-ist-analyse}
\import{./}{05-soll-konzept}
\import{./}{06-implementierung}
\import{./}{07-evaluation}
\import{./}{08-diskussion}
\import{./}{09-zusammenfassung}
%%
%%%%%%%%%%%%%%%%%%%%%%%%%%%%%%%%%%%%%%%%%%%%%%%%%%%%%%%%%%%%%%%%%%%%%%%%%%%%%%%%

%%%%%%%%%%%%%%%%%%%%%%%%%%%%%%%%%%%%%%%%%%%%%%%%%%%%%%%%%%%%%%%%%%%%%%%%%%%%%%%%
%%
%% Print the bibliography
%%
\cleardoubleoddpage
\printbibliography[title={Literaturverzeichnis}]
%%
%%%%%%%%%%%%%%%%%%%%%%%%%%%%%%%%%%%%%%%%%%%%%%%%%%%%%%%%%%%%%%%%%%%%%%%%%%%%%%%%

%% Begin with the appendix part (all further chapters will be appendices)
\appendix

%%%%%%%%%%%%%%%%%%%%%%%%%%%%%%%%%%%%%%%%%%%%%%%%%%%%%%%%%%%%%%%%%%%%%%%%%%%%%%%%
%%
%% Include your appendix chapters
%%
\import{./}{91-anhang}
%%
%%%%%%%%%%%%%%%%%%%%%%%%%%%%%%%%%%%%%%%%%%%%%%%%%%%%%%%%%%%%%%%%%%%%%%%%%%%%%%%%

\cleardoubleoddpage

\end{document}
\endinput
