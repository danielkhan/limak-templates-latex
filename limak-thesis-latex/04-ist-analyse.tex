% !TeX root = main-limak-thesis.tex

%%%%%%%%%%%%%%%%%%%%%%%%%%%%%%%%%%%%%%%%%%%%%%%%%%%%%%%%%%%%%%%%%%%%%%%%%%%%%%%%
%% KAPITEL 4: EMPIRISCHE ANALYSE / IST-ZUSTAND
%%
%% Gemäß LIMAK Leitfaden: "Empirische und/oder konzeptionelle Ergebnisse"
%%
%% Dieses Kapitel kann je nach Thema unterschiedlich benannt werden:
%% - Ist-Analyse (bei Prozessoptimierung)
%% - Empirische Ergebnisse (bei Umfragen/Interviews)
%% - Fallbeschreibung (bei Case Studies)
%% - Status Quo Analyse
%%%%%%%%%%%%%%%%%%%%%%%%%%%%%%%%%%%%%%%%%%%%%%%%%%%%%%%%%%%%%%%%%%%%%%%%%%%%%%%%

\chapter{Analyse des Ist-Zustands}
\label{chap:ist-analyse}

%% TODO: Passen Sie Kapiteltitel und Inhalt an Ihr Thema an

\section{Beschreibung des Untersuchungskontexts}
\label{sec:unternehmenskontext}

%% Beschreiben Sie hier den Kontext Ihrer Untersuchung

[Unternehmen/Organisation] ist ein [Beschreibung] mit [Größe/Standorte]. Das Unternehmen beschäftigt insgesamt [Anzahl] Mitarbeitende und ist in [Branche/Markt] tätig.

\subsection{Organisationsstruktur}

Die Struktur umfasst folgende Bereiche:
\begin{itemize}
    \item {[Bereich 1]}
    \item {[Bereich 2]}
    \item {[Bereich 3]}
    \item {[Bereich 4]}
\end{itemize}

\subsection{Ausgangssituation}

Die aktuelle Situation ist geprägt durch:
\begin{itemize}
    \item {[Merkmal 1]}
    \item {[Merkmal 2]}
    \item {[Merkmal 3]}
\end{itemize}

\section{Analyse der aktuellen Situation}
\label{sec:ist-analyse-ergebnisse}

%% Präsentieren Sie hier Ihre Analyseergebnisse

\subsection{[Analysebereich 1]}
\label{subsec:bereich1-ist}

\textbf{Ausgangssituation:}
[Beschreibung der aktuellen Situation in diesem Bereich]

\textbf{Beobachtete Merkmale:}
\begin{enumerate}
    \item {[Merkmal/Prozessschritt 1]}
    \item {[Merkmal/Prozessschritt 2]}
    \item {[Merkmal/Prozessschritt 3]}
\end{enumerate}

\textbf{Identifizierte Probleme/Herausforderungen:}
\begin{itemize}
    \item {[Problem 1]}
    \item {[Problem 2]}
    \item {[Problem 3]}
\end{itemize}

\subsection{[Analysebereich 2]}
\label{subsec:bereich2-ist}

[Analog zu Bereich 1 strukturieren]

\subsection{[Analysebereich 3]}
\label{subsec:bereich3-ist}

[Analog zu Bereich 1 strukturieren]

\section{Erhebung der Ausgangskennzahlen}
\label{sec:ist-kpis}

%% Präsentieren Sie quantitative Daten zum Ist-Zustand

\subsection{Kennzahlen Übersicht}

\begin{table}[htbp]
\centering
\begin{tabular}{@{}lc@{}}
\toprule
\textbf{Kennzahl} & \textbf{Aktueller Wert (IST)} \\
\midrule
{[Kennzahl 1]} & {[Wert]} \\
{[Kennzahl 2]} & {[Wert]} \\
{[Kennzahl 3]} & {[Wert]} \\
{[Kennzahl 4]} & {[Wert]} \\
\bottomrule
\end{tabular}
\caption{Übersicht der erhobenen Ist-Kennzahlen}
\label{tab:ist-kennzahlen}
\end{table}

\subsection{Detailanalyse}

[Detaillierte Beschreibung und Interpretation der Kennzahlen]

\section{Zusammenfassung der Analyseergebnisse}
\label{sec:ist-zusammenfassung}

%% Fassen Sie die wichtigsten Erkenntnisse zusammen

Die Analyse zeigt folgende Haupterkenntnisse:
\begin{enumerate}
    \item \textbf{[Erkenntnis 1]:} [Beschreibung]
    \item \textbf{[Erkenntnis 2]:} [Beschreibung]
    \item \textbf{[Erkenntnis 3]:} [Beschreibung]
    \item \textbf{[Erkenntnis 4]:} [Beschreibung]
\end{enumerate}

Diese Erkenntnisse bilden die Grundlage für [nächsten Schritt, z.B. Konzeptentwicklung, Handlungsempfehlungen] im folgenden Kapitel.
