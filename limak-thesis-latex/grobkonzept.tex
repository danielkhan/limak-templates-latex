% !TeX program = xelatex
% !TeX encoding = UTF-8
% !TeX spellcheck = de_DE
%%%%%%%%%%%%%%%%%%%%%%%%%%%%%%%%%%%%%%%%%%%%%%%%%%%%%%%%%%%%%%%%%%%%%%%%%%%%%%%%
%%
%% LIMAK Grobkonzept Vorlage
%%
%% Das Grobkonzept ist ein kurzes Dokument (ca. 2-4 Seiten), das vor Beginn
%% der Masterarbeit eingereicht wird. Es dient zur Abstimmung des Themas
%% mit dem Betreuer.
%%
%%%%%%%%%%%%%%%%%%%%%%%%%%%%%%%%%%%%%%%%%%%%%%%%%%%%%%%%%%%%%%%%%%%%%%%%%%%%%%%%

\documentclass[a4paper,oneside,11pt,ngerman]{scrartcl}

% Pakete
\usepackage[utf8]{inputenc}
\usepackage[ngerman]{babel}
\usepackage[T1]{fontenc}
\usepackage{fontspec}
\usepackage{graphicx}
\usepackage{xcolor}
\usepackage{hyperref}
\usepackage{enumitem}
\usepackage{booktabs}
\usepackage{geometry}
\usepackage{titlesec}

% Seitenraender (entsprechend LIMAK Word-Vorlage)
\geometry{
    a4paper,
    left=3cm,
    right=2.5cm,
    top=3cm,
    bottom=3cm
}

% 1.5 Zeilenabstand (LIMAK Standard)
\usepackage{setspace}
\onehalfspacing

% Schriftarten - Arial (entsprechend LIMAK Word-Vorlage)
\setmainfont{Arial}
\setsansfont{Arial}

% LIMAK Farben (Business School)
\definecolor{limakblue}{RGB}{0,51,153}
\definecolor{limakgray}{RGB}{102,102,102}

% Hyperref Setup
%% TODO: Passen Sie die Metadaten an
\hypersetup{
    colorlinks=true,
    linkcolor=limakblue,
    urlcolor=limakblue,
    citecolor=limakblue,
    pdfauthor={Ihr Name},
    pdftitle={Grobkonzept Masterarbeit - LIMAK},
    pdfsubject={Kurze Beschreibung des Themas}
}

% Titel-Formatierung
\titleformat{\section}
  {\normalfont\Large\bfseries\color{limakblue}}{\thesection}{1em}{}
\titleformat{\subsection}
  {\normalfont\large\bfseries\color{limakgray}}{\thesubsection}{1em}{}

\begin{document}

% Titelseite mit LIMAK Logo
\begin{titlepage}
    \begin{center}
        \includegraphics[height=20mm]{logos/jku_LIMAK_black.png}

        \vspace{2cm}

        {\Huge\bfseries\color{limakblue} Grobkonzept\\[0.3em] Masterarbeit}

        \vspace{1.5cm}

        %% TODO: Titel der Arbeit anpassen
        {\Large\bfseries Titel der geplanten Masterarbeit\\[0.5em]
        Optionaler Untertitel}

        \vspace{1cm}


        \vfill

        {\large
        %% TODO: Anpassen Sie diese Angaben
        \textbf{Verfasser:} Ihr Name\\[0.5em]
        \textbf{Studiengang:} z.B. EMBA fuer General Management\\[0.5em]
        \textbf{Institution:} LIMAK Austrian Business School\\[0.5em]
        \textbf{Datum:} \today
        }

        \vspace{1cm}
    \end{center}
\end{titlepage}

%%%%%%%%%%%%%%%%%%%%%%%%%%%%%%%%%%%%%%%%%%%%%%%%%%%%%%%%%%%%%%%%%%%%%%%%%%%%%%%%
%% INHALT DES GROBKONZEPTS
%%%%%%%%%%%%%%%%%%%%%%%%%%%%%%%%%%%%%%%%%%%%%%%%%%%%%%%%%%%%%%%%%%%%%%%%%%%%%%%%

\section{Ausgangslage, Problemstellung und Motivation}

%% Beschreiben Sie den Kontext und die Ausgangssituation
Beschreibung des Unternehmens/der Organisation und der aktuellen Situation. Welches Problem oder welche Herausforderung besteht? Warum ist dieses Thema relevant?

\subsection{Problemstellung}

Konkrete Beschreibung des Problems, das in der Arbeit behandelt werden soll. Was funktioniert nicht optimal? Welche Luecken bestehen?

\begin{itemize}[leftmargin=*]
    \item Aspekt 1 der Problemstellung
    \item Aspekt 2 der Problemstellung
    \item Aspekt 3 der Problemstellung
\end{itemize}

\subsection{Motivation}

Warum ist dieses Thema fuer Sie persoenlich und/oder fuer das Unternehmen wichtig? Was erhoffen Sie sich von der Bearbeitung?

\section{Ziele der Arbeit}

%% Formulieren Sie klare, messbare Ziele
\begin{itemize}[leftmargin=*]
    \item \textbf{Deskriptives Ziel:} z.B. Entwicklung eines Modells, Darstellung von Best Practices
    \item \textbf{Evaluatives Ziel:} z.B. Bewertung von Alternativen, Messung von Effekten
    \item \textbf{Praktisches Ziel:} z.B. Implementierung einer Loesung, Entwicklung von Handlungsempfehlungen
    \item \textbf{Wissenschaftliches Ziel:} z.B. Beitrag zur Forschung, Schliessen einer Forschungsluecke
\end{itemize}

\section{Grundsaetzliche Gliederung der Arbeit}

\subsection{Geplante Kapitelstruktur}

%% Skizzieren Sie die voraussichtliche Struktur
\begin{enumerate}[leftmargin=*]
    \item \textbf{Einleitung}
    \begin{itemize}
        \item Ausgangssituation und Problemstellung
        \item Zielsetzung und Forschungsfrage
        \item Aufbau der Arbeit
    \end{itemize}

    \item \textbf{Theoretische Grundlagen}
    \begin{itemize}
        \item Theoriebereich 1
        \item Theoriebereich 2
        \item Stand der Forschung
    \end{itemize}

    \item \textbf{Methodik}
    \begin{itemize}
        \item Forschungsdesign
        \item Datenerhebung
        \item Datenauswertung
    \end{itemize}

    \item \textbf{Empirischer Teil / Analyse}
    \begin{itemize}
        \item Ist-Analyse / Fallstudie / Erhebung
        \item Ergebnisdarstellung
    \end{itemize}

    \item \textbf{Konzept / Loesung}
    \begin{itemize}
        \item Entwicklung des Loesungsansatzes
        \item Implementierung / Umsetzung
    \end{itemize}

    \item \textbf{Evaluation / Diskussion}
    \begin{itemize}
        \item Bewertung der Ergebnisse
        \item Limitationen
    \end{itemize}

    \item \textbf{Zusammenfassung und Ausblick}
    \begin{itemize}
        \item Beantwortung der Forschungsfrage
        \item Praktische Implikationen
        \item Weitere Forschungsbedarfe
    \end{itemize}
\end{enumerate}

\subsection{Inhaltliche Prioritaeten}

%% Was sind die Schwerpunkte Ihrer Arbeit?
\begin{itemize}[leftmargin=*]
    \item \textbf{Schwerpunkt 1:} Kurze Beschreibung
    \item \textbf{Schwerpunkt 2:} Kurze Beschreibung
    \item \textbf{Schwerpunkt 3:} Kurze Beschreibung
\end{itemize}

\section{Methodik (optional)}

%% Falls bereits bekannt, skizzieren Sie das methodische Vorgehen
Kurze Beschreibung der geplanten Forschungsmethodik: Qualitativ/Quantitativ? Fallstudie? Interviews? Fragebogen? etc.

\section{Zeitplan (optional)}

%% Grober Zeitplan fuer die Erstellung der Arbeit
Skizzieren Sie wichtige Meilensteine und den geplanten Abgabetermin.

\end{document}
