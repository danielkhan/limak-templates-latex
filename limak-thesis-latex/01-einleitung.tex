% !TeX root = main-limak-thesis.tex

%%%%%%%%%%%%%%%%%%%%%%%%%%%%%%%%%%%%%%%%%%%%%%%%%%%%%%%%%%%%%%%%%%%%%%%%%%%%%%%%
%% KAPITEL 1: EINLEITUNG
%%
%% Gemäß LIMAK Leitfaden soll die Einleitung enthalten:
%% - Problemstellung bzw. Motivation sowie Ziele und Struktur der Arbeit
%%
%% Tipp: Die Einleitung bildet den "roten Faden" für die gesamte Arbeit
%%%%%%%%%%%%%%%%%%%%%%%%%%%%%%%%%%%%%%%%%%%%%%%%%%%%%%%%%%%%%%%%%%%%%%%%%%%%%%%%

\chapter{Einleitung}
\label{chap:einleitung}

%% TODO: Ersetzen Sie alle Platzhalter [...] mit Ihren eigenen Inhalten

\section{Ausgangssituation und Motivation}
\label{sec:ausgangssituation}

%% Beschreiben Sie hier den Kontext und die Ausgangslage Ihres Themas.
%% Bei praxisorientierten Arbeiten: Stellen Sie das Unternehmen und die
%% relevante Branchensituation vor.

[Unternehmen/Organisation] ist ein [Branche]-Unternehmen mit [Größe/Standorte]. Das Unternehmen steht vor der Herausforderung, [aktuelle Situation/Problemlage]. Diese Situation ist typisch für [Branche/Unternehmensgröße] und hat direkte Auswirkungen auf [relevante Geschäftsbereiche].

Die Motivation für diese Arbeit ergibt sich aus [konkreter Anlass], da [Begründung der Dringlichkeit]. Für die Praxis ist diese Fragestellung von hoher Relevanz, weil [praktische Bedeutung] \parencite{Beispielquelle2024}.

\section{Problemstellung}
\label{sec:problemstellung}

%% Formulieren Sie hier klar und präzise das Problem, das Sie in Ihrer
%% Arbeit bearbeiten möchten.

Die zentrale Herausforderung besteht darin, [Kernproblem]. Dabei ergeben sich folgende spezifische Fragen:

\begin{itemize}
    \item Wie kann [Teilproblem 1] systematisch erfasst und analysiert werden?
    \item Welche [Faktoren/Methoden/Ansätze] eignen sich für [Teilproblem 2]?
    \item Wie lassen sich [gewünschte Ergebnisse] messen und bewerten?
    \item Welche Besonderheiten ergeben sich durch [Kontext/Rahmenbedingungen]?
\end{itemize}

\section{Zielsetzung der Arbeit}
\label{sec:zielsetzung}

%% Formulieren Sie klare, messbare Ziele. Unterscheiden Sie zwischen
%% wissenschaftlichen und praktischen Zielen.

Diese Master Thesis verfolgt mehrere Zielsetzungen:

\begin{enumerate}
    \item \textbf{Wissenschaftliches Ziel:} [z.B. Entwicklung eines Modells/Frameworks, Überprüfung einer Theorie, Schließen einer Forschungslücke]

    \item \textbf{Praktisches Ziel:} [z.B. Entwicklung von Handlungsempfehlungen, Implementierung einer Lösung, Optimierung von Prozessen]

    \item \textbf{Evaluatives Ziel:} [z.B. Bewertung der Wirksamkeit, Messung von Verbesserungen, Validierung eines Konzepts]
\end{enumerate}

\section{Forschungsfrage}
\label{sec:forschungsfrage}

%% Die Forschungsfrage sollte präzise, beantwortbar und relevant sein.
%% Sie kann als Hauptfrage mit Unterfragen strukturiert werden.

Die zentrale Forschungsfrage lautet:

\begin{quote}
\textit{Wie beeinflusst [untersuchte Variable/Maßnahme] die [Zielgröße] bei/in [Kontext] -- und welche [Faktoren/Bedingungen] sind dabei besonders relevant?}
\end{quote}

Zur Beantwortung dieser Hauptfrage werden folgende Unterfragen bearbeitet:
\begin{enumerate}
    \item {[Unterfrage 1]?}
    \item {[Unterfrage 2]?}
    \item {[Unterfrage 3]?}
\end{enumerate}

\section{Aufbau der Arbeit}
\label{sec:aufbau}

%% Geben Sie einen Überblick über die Struktur Ihrer Arbeit.
%% Dies hilft den Leser*innen bei der Orientierung.

Die Arbeit gliedert sich wie folgt:

\textbf{Kapitel 2} behandelt die theoretischen Grundlagen zu [Theoriebereich 1] und [Theoriebereich 2]. Es werden die relevanten Konzepte und der aktuelle Forschungsstand dargestellt.

\textbf{Kapitel 3} erläutert die methodische Vorgehensweise, insbesondere [gewählte Methoden] und deren Begründung.

\textbf{Kapitel 4} präsentiert die Ergebnisse der [Art der Analyse, z.B. Ist-Analyse, empirischen Untersuchung].

\textbf{Kapitel 5} beschreibt [nächsten Arbeitsschritt, z.B. Konzeptentwicklung, Lösungsansatz].

\textbf{Kapitel 6} dokumentiert [z.B. Implementierung, Umsetzung, Anwendung].

\textbf{Kapitel 7} stellt die Evaluationsergebnisse dar und bewertet [Zielgröße].

\textbf{Kapitel 8} diskutiert die Ergebnisse im wissenschaftlichen und praktischen Kontext.

\textbf{Kapitel 9} fasst die Arbeit zusammen und gibt einen Ausblick auf weitere Entwicklungen.
