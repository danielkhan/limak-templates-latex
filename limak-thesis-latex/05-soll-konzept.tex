% !TeX root = main-limak-thesis.tex

%%%%%%%%%%%%%%%%%%%%%%%%%%%%%%%%%%%%%%%%%%%%%%%%%%%%%%%%%%%%%%%%%%%%%%%%%%%%%%%%
%% KAPITEL 5: KONZEPTENTWICKLUNG / SOLL-ZUSTAND
%%
%% Dieses Kapitel kann je nach Thema unterschiedlich benannt werden:
%% - Soll-Konzept (bei Prozessoptimierung)
%% - Lösungskonzept
%% - Strategieentwicklung
%% - Handlungsempfehlungen
%%%%%%%%%%%%%%%%%%%%%%%%%%%%%%%%%%%%%%%%%%%%%%%%%%%%%%%%%%%%%%%%%%%%%%%%%%%%%%%%

\chapter{Konzeptentwicklung}
\label{chap:soll-konzept}

%% TODO: Passen Sie Kapiteltitel und Inhalt an Ihr Thema an

\section{Zielsetzungen}
\label{sec:konzept-ziele}

%% Definieren Sie klare, messbare Ziele für Ihr Konzept

Basierend auf den Erkenntnissen der Analyse werden folgende Ziele definiert:

\begin{itemize}
    \item \textbf{Ziel 1:} [Konkrete, messbare Zielbeschreibung]
    \item \textbf{Ziel 2:} [Konkrete, messbare Zielbeschreibung]
    \item \textbf{Ziel 3:} [Konkrete, messbare Zielbeschreibung]
    \item \textbf{Ziel 4:} [Konkrete, messbare Zielbeschreibung]
\end{itemize}

\section{Konzeptionelle Grundlagen}
\label{sec:konzept-grundlagen}

%% Erläutern Sie die theoretische Basis Ihres Konzepts

Das entwickelte Konzept basiert auf [theoretische Grundlage/Best Practices/Framework] und berücksichtigt die spezifischen Anforderungen von [Kontext].

\subsection{Leitprinzipien}

Folgende Leitprinzipien liegen dem Konzept zugrunde:
\begin{enumerate}
    \item {[Prinzip 1 mit Begründung]}
    \item {[Prinzip 2 mit Begründung]}
    \item {[Prinzip 3 mit Begründung]}
\end{enumerate}

\section{Lösungskonzept}
\label{sec:loesungskonzept}

\subsection{[Lösungsbereich 1]}
\label{subsec:loesung1}

\textbf{Zielsetzung:}
[Was soll in diesem Bereich erreicht werden?]

\textbf{Vorgeschlagene Maßnahmen:}
\begin{enumerate}
    \item {[Maßnahme 1]}
    \item {[Maßnahme 2]}
    \item {[Maßnahme 3]}
\end{enumerate}

\textbf{Erwartete Wirkung:}
\begin{itemize}
    \item {[Wirkung 1]}
    \item {[Wirkung 2]}
\end{itemize}

\subsection{[Lösungsbereich 2]}
\label{subsec:loesung2}

[Analog zu Lösungsbereich 1 strukturieren]

\subsection{[Lösungsbereich 3]}
\label{subsec:loesung3}

[Analog zu Lösungsbereich 1 strukturieren]

\section{Bewertung der Alternativen}
\label{sec:alternativen}

%% Falls Sie verschiedene Lösungsansätze verglichen haben

\begin{table}[htbp]
\centering
\begin{tabular}{@{}lccc@{}}
\toprule
\textbf{Kriterium} & \textbf{Alternative A} & \textbf{Alternative B} & \textbf{Alternative C} \\
\midrule
{[Kriterium 1]} & {[Bewertung]} & {[Bewertung]} & {[Bewertung]} \\
{[Kriterium 2]} & {[Bewertung]} & {[Bewertung]} & {[Bewertung]} \\
{[Kriterium 3]} & {[Bewertung]} & {[Bewertung]} & {[Bewertung]} \\
\midrule
\textbf{Gesamtbewertung} & {[Summe]} & {[Summe]} & {[Summe]} \\
\bottomrule
\end{tabular}
\caption{Vergleich der Lösungsalternativen}
\label{tab:alternativen-vergleich}
\end{table}

\section{Erwartete Ergebnisse}
\label{sec:erwartete-ergebnisse}

\begin{table}[htbp]
\centering
\begin{tabular}{@{}lccc@{}}
\toprule
\textbf{Kennzahl} & \textbf{IST} & \textbf{SOLL} & \textbf{Erwartete Verbesserung} \\
\midrule
{[Kennzahl 1]} & {[Wert]} & {[Zielwert]} & {[+/-X\%]} \\
{[Kennzahl 2]} & {[Wert]} & {[Zielwert]} & {[+/-X\%]} \\
{[Kennzahl 3]} & {[Wert]} & {[Zielwert]} & {[+/-X\%]} \\
\bottomrule
\end{tabular}
\caption{Erwartete Verbesserungen durch das Konzept}
\label{tab:erwartete-verbesserungen}
\end{table}
