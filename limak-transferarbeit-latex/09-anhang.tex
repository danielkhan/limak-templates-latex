% !TeX root = main-transferarbeit.tex

%%%%%%%%%%%%%%%%%%%%%%%%%%%%%%%%%%%%%%%%%%%%%%%%%%%%%%%%%%%%%%%%%%%%%%%%%%%%%%%%
%% ANHANG (OPTIONAL)
%%
%% Der Anhang enthält ergänzende Materialien zur Transferarbeit.
%%
%% Typische Inhalte:
%% - Detaillierte Analysen und Auswertungen
%% - Fragebögen und Interviewleitfäden
%% - Zusätzliche Abbildungen und Tabellen
%% - Rohdaten und Dokumentationen
%%%%%%%%%%%%%%%%%%%%%%%%%%%%%%%%%%%%%%%%%%%%%%%%%%%%%%%%%%%%%%%%%%%%%%%%%%%%%%%%

\section{Anhang}
\label{sec:anhang}

%% TODO: Fügen Sie hier bei Bedarf ergänzende Materialien ein.
%% Aktivieren Sie den Anhang in der main-transferarbeit.tex, indem Sie
%% die entsprechenden Zeilen am Ende der Datei einkommentieren.

\subsection{[Titel des Anhangs A]}
\label{subsec:anhang-a}

[Inhalt des ersten Anhangs]

%% Beispiel für eine Abbildung:
%% \begin{figure}[htbp]
%% \centering
%% \includegraphics[width=0.85\textwidth]{dateiname.png}
%% \caption{[Beschreibung der Abbildung]}
%% \label{fig:anhang-abbildung}
%% \end{figure}

\subsection{[Titel des Anhangs B]}
\label{subsec:anhang-b}

[Inhalt des zweiten Anhangs]

%% Beispiel für eine Tabelle:
%% \begin{table}[htbp]
%% \centering
%% \begin{tabular}{@{}lll@{}}
%% \toprule
%% \textbf{Spalte 1} & \textbf{Spalte 2} & \textbf{Spalte 3} \\
%% \midrule
%% {[Wert 1]} & {[Wert 2]} & {[Wert 3]} \\
%% {[Wert 4]} & {[Wert 5]} & {[Wert 6]} \\
%% \bottomrule
%% \end{tabular}
%% \caption{[Beschreibung der Tabelle]}
%% \label{tab:anhang-tabelle}
%% \end{table}

