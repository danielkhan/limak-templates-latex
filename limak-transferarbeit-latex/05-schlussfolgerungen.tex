% !TeX root = main-transferarbeit.tex

%%%%%%%%%%%%%%%%%%%%%%%%%%%%%%%%%%%%%%%%%%%%%%%%%%%%%%%%%%%%%%%%%%%%%%%%%%%%%%%%
%% ABSCHNITT 5: SCHLUSSFOLGERUNGEN MIT HANDLUNGSEMPFEHLUNGEN
%%
%% Gemäß LIMAK Leitfaden:
%% Fazit der Arbeit mit konkreten Handlungsempfehlungen für die Praxis.
%%
%% Tipps:
%% - Zusammenfassung der wichtigsten Erkenntnisse
%% - Beantwortung der Fragestellung
%% - Konkrete, umsetzbare Handlungsempfehlungen
%% - Reflexion des Transferprozesses
%% - Ausblick auf nächste Schritte
%%%%%%%%%%%%%%%%%%%%%%%%%%%%%%%%%%%%%%%%%%%%%%%%%%%%%%%%%%%%%%%%%%%%%%%%%%%%%%%%

\section{Schlussfolgerungen mit Handlungsempfehlungen}
\label{sec:schlussfolgerungen}

%% TODO: Ersetzen Sie alle Platzhalter [...] mit Ihren eigenen Inhalten.

\subsection{Zusammenfassung der Erkenntnisse}

Die Transferarbeit hat sich mit der Fragestellung [Kurzform der Fragestellung] befasst. Die wichtigsten Erkenntnisse lassen sich wie folgt zusammenfassen:

\begin{enumerate}
    \item \textbf{[Erkenntnis 1]:} [Kurze Beschreibung]
    \item \textbf{[Erkenntnis 2]:} [Kurze Beschreibung]
    \item \textbf{[Erkenntnis 3]:} [Kurze Beschreibung]
\end{enumerate}

\subsection{Beantwortung der Fragestellung}

Die zentrale Fragestellung kann wie folgt beantwortet werden:

[Formulieren Sie hier eine prägnante Antwort auf Ihre Fragestellung, die die wichtigsten Erkenntnisse integriert.]

\subsection{Handlungsempfehlungen}

Aus den Erkenntnissen der Transferarbeit werden folgende konkrete Handlungsempfehlungen abgeleitet:

\subsubsection{Handlungsempfehlung 1: [Titel]}

\textbf{Was:} [Beschreibung der Maßnahme]

\textbf{Warum:} [Begründung basierend auf den Erkenntnissen]

\textbf{Wie:} [Konkrete Umsetzungsschritte]

\textbf{Wann:} [Zeitrahmen für die Umsetzung]

\subsubsection{Handlungsempfehlung 2: [Titel]}

\textbf{Was:} [Beschreibung der Maßnahme]

\textbf{Warum:} [Begründung]

\textbf{Wie:} [Umsetzungsschritte]

\textbf{Wann:} [Zeitrahmen]

\subsubsection{Handlungsempfehlung 3: [Titel]}

\textbf{Was:} [Beschreibung der Maßnahme]

\textbf{Warum:} [Begründung]

\textbf{Wie:} [Umsetzungsschritte]

\textbf{Wann:} [Zeitrahmen]

\subsection{Priorisierung der Maßnahmen}

%% Übersichtstabelle zur Priorisierung

\begin{table}[htbp]
\centering
\begin{tabular}{@{}lccc@{}}
\toprule
\textbf{Handlungsempfehlung} & \textbf{Priorität} & \textbf{Aufwand} & \textbf{Wirkung} \\
\midrule
{[Empfehlung 1]} & {[1/2/3]} & {[hoch/mittel/niedrig]} & {[hoch/mittel/niedrig]} \\
{[Empfehlung 2]} & {[1/2/3]} & {[hoch/mittel/niedrig]} & {[hoch/mittel/niedrig]} \\
{[Empfehlung 3]} & {[1/2/3]} & {[hoch/mittel/niedrig]} & {[hoch/mittel/niedrig]} \\
\bottomrule
\end{tabular}
\caption{Priorisierung der Handlungsempfehlungen}
\label{tab:priorisierung}
\end{table}

\subsection{Reflexion des Theorie-Praxis-Transfers}

[Reflektieren Sie den Transfer von der Theorie in Ihre Praxis. Was hat gut funktioniert? Wo gab es Herausforderungen?]

Die Anwendung der theoretischen Konzepte auf die Praxissituation hat gezeigt, dass [Erkenntnisse zum Transfer]. Besonders hilfreich war [Aspekt], während [anderer Aspekt] in der praktischen Anwendung angepasst werden musste.

\subsection{Persönliche Learnings}

[Beschreiben Sie Ihre persönlichen Lerneffekte aus der Bearbeitung der Transferarbeit.]

\subsection{Ausblick und nächste Schritte}

Als nächste konkrete Schritte sind geplant:

\begin{enumerate}
    \item {[Nächster Schritt 1 mit Zeitrahmen]}
    \item {[Nächster Schritt 2 mit Zeitrahmen]}
    \item {[Nächster Schritt 3 mit Zeitrahmen]}
\end{enumerate}

[Optionaler Abschlussgedanke oder Fazit]

