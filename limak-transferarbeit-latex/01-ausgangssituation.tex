% !TeX root = main-transferarbeit.tex

%%%%%%%%%%%%%%%%%%%%%%%%%%%%%%%%%%%%%%%%%%%%%%%%%%%%%%%%%%%%%%%%%%%%%%%%%%%%%%%%
%% ABSCHNITT 1: AUSGANGSSITUATION
%%
%% Gemäß LIMAK Leitfaden:
%% Beschreibung des Kontexts und der Ausgangslage für die Transferarbeit.
%%
%% Tipps:
%% - Kontext des Unternehmens/der Organisation beschreiben
%% - Relevante Rahmenbedingungen darstellen
%% - Aktuelle Situation und Herausforderungen skizzieren
%% - Bezug zur Lehrveranstaltung herstellen
%%%%%%%%%%%%%%%%%%%%%%%%%%%%%%%%%%%%%%%%%%%%%%%%%%%%%%%%%%%%%%%%%%%%%%%%%%%%%%%%

\section{Ausgangssituation}
\label{sec:ausgangssituation}

%% TODO: Ersetzen Sie alle Platzhalter [...] mit Ihren eigenen Inhalten.

\subsection{Kontext und Hintergrund}

[Beschreiben Sie den Kontext, in dem Ihre Transferarbeit angesiedelt ist. Dies kann Ihr Unternehmen, Ihre Abteilung oder ein spezifisches Projekt sein.]

[Beispiel: Das Unternehmen [Name] ist ein mittelständisches Unternehmen im Bereich [Branche] mit [X] Mitarbeitenden und einem Jahresumsatz von [Y] Euro. Das Unternehmen steht vor der Herausforderung, [zentrale Herausforderung].]

\subsection{Aktuelle Situation}

[Beschreiben Sie die aktuelle Situation, die den Anlass für diese Transferarbeit bildet.]

Die aktuelle Situation ist gekennzeichnet durch:
\begin{itemize}
    \item {[Merkmal/Herausforderung 1]}
    \item {[Merkmal/Herausforderung 2]}
    \item {[Merkmal/Herausforderung 3]}
\end{itemize}

\subsection{Relevanz für die berufliche Praxis}

[Erläutern Sie, warum das Thema für Ihre berufliche Praxis relevant ist und welchen Mehrwert eine systematische Bearbeitung bringt.]

\subsection{Bezug zur Lehrveranstaltung}

[Stellen Sie den Bezug zur Lehrveranstaltung her. Welche Konzepte, Modelle oder Theorien aus dem Kurs sind für Ihre Ausgangssituation relevant?]

Die in der Lehrveranstaltung ``[Name der Lehrveranstaltung]'' behandelten Konzepte zu [Themenbereich] sind für diese Situation besonders relevant, da [Begründung].

