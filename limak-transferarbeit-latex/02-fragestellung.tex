% !TeX root = main-transferarbeit.tex

%%%%%%%%%%%%%%%%%%%%%%%%%%%%%%%%%%%%%%%%%%%%%%%%%%%%%%%%%%%%%%%%%%%%%%%%%%%%%%%%
%% ABSCHNITT 2: FRAGESTELLUNG
%%
%% Gemäß LIMAK Leitfaden:
%% Formulierung der zentralen Frage, die in der Transferarbeit bearbeitet wird.
%%
%% Tipps:
%% - Klare, präzise Formulierung der Hauptfrage
%% - Ggf. Unterfragen zur Strukturierung
%% - Abgrenzung: Was wird behandelt, was nicht?
%% - Die Fragestellung sollte praxisorientiert sein
%%%%%%%%%%%%%%%%%%%%%%%%%%%%%%%%%%%%%%%%%%%%%%%%%%%%%%%%%%%%%%%%%%%%%%%%%%%%%%%%

\section{Fragestellung}
\label{sec:fragestellung}

%% TODO: Ersetzen Sie alle Platzhalter [...] mit Ihren eigenen Inhalten.

\subsection{Zentrale Fragestellung}

Aus der beschriebenen Ausgangssituation ergibt sich folgende zentrale Fragestellung:

\begin{quote}
\textit{[Formulieren Sie hier Ihre zentrale Fragestellung. Diese sollte klar, präzise und praxisorientiert sein.]}
\end{quote}

%% Beispiel für eine Fragestellung:
%% ``Wie kann das Change Management bei der Einführung eines neuen ERP-Systems
%% im Unternehmen [Name] so gestaltet werden, dass die Mitarbeiterakzeptanz
%% maximiert und der Implementierungserfolg sichergestellt wird?''

\subsection{Teilfragen}

Zur Beantwortung der zentralen Fragestellung werden folgende Teilfragen bearbeitet:

\begin{enumerate}
    \item {[Teilfrage 1: z.B. theoretische Grundlage]}
    \item {[Teilfrage 2: z.B. Analyse der aktuellen Situation]}
    \item {[Teilfrage 3: z.B. Entwicklung von Lösungsansätzen]}
    \item {[Teilfrage 4: z.B. Umsetzungsempfehlungen]}
\end{enumerate}

\subsection{Abgrenzung}

[Grenzen Sie den Umfang Ihrer Arbeit ab. Was wird behandelt, was wird explizit nicht behandelt?]

Die Transferarbeit konzentriert sich auf [Fokus] und behandelt nicht [Abgrenzung]. Diese Einschränkung ist notwendig, da [Begründung].

